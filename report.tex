%% TODO : questions
%% TODO : euh, problème de cohérence. Comme v doit etre réel, ça serait bien de passer tout les état en réel et pas des entiers...

\documentclass[a4paper,12pt]{article}
\usepackage[T1]{fontenc}
\usepackage[utf8]{inputenc}
\usepackage[margin=1in]{geometry}
\usepackage{lmodern}
\usepackage{amssymb}
\usepackage{stmaryrd}
\usepackage{amsmath}
\usepackage{bbm}
\usepackage{url}
\usepackage[english]{babel}

\usepackage{hyperref}


\title{Parallel and Distrubuted Algorithms and Programs Project}
\author{Raphaël Monat}

\begin{document}

\maketitle

\section{Euler's approximation of differential equations}

\paragraph{Question 1a)}
Let's study $y' = ay$, with $a \in \mathbb{R}$. Using Euler's approximation, we get $y(t + dt) = y(t) + y'(t)dt = y(t) + a y(t) dt$, ie: $y(t + dt) = y(t) \cdot (1 + a \cdot dt)$. If we choose $a = -3$ and $dt = 1$, we have: $y(t+1) = - y(t)$. The solution to the differential equation is $y(t) = y(0) \cdot e^{-3x}$. The approximated solution is $y(k) = (-2)^{k+1}y(0)$ for any $k \in \mathbb{N}$. This is especially unacceptable, as even the asymptotic behavior is not the same.

\paragraph{Question 1b)} To compute the next step of an approximation of a differential equation of order n, we need to store the values of $u(t), ..., u^{(n-1)}(t)$, to be able to compute $u(t+dt)$: we need to store $n$ values.

\section{Cellular automata}

\paragraph{Question 2a)} We need to update every cell of the grid, so we need to call $\delta$ on each cell of this grid: $N \times M$ applications of $\delta$ to compute $X^t$ from $X^{t-1}$. To compute $X^t$ from $X^0$, we need $t \times NM$ applications of $\delta$.

\paragraph{Question 2b)} Suppose that we have a $p \times q$ 2D grid, and suppose also that $p \mid N$, $q \mid M$ for the sake of simplicity. Let $(i, j) \in \llbracket 0; p-1 \rrbracket \times \llbracket 0; q - 1 \rrbracket$. On the machine of coordinates $(i, j)$, we store the subgrid $\mathcal{G}_{i,j}$, of size $\frac{N}{p} \times \frac{M}{q}$, containing the following elements: $G_{k, l}$ with $(k, l) \in \llbracket i\frac{N}{p}; (i+1)\frac{N}{p}-1 \rrbracket \times \llbracket j\frac{M}{q}; (j+1)\frac{M}{q}-1 \rrbracket$. On each machine, we can localy update every element of the local grid, except the edges of the grid. To compute the edges of the grid, we need to get one edge of each grid $G_{i\pm1, j\pm1}$ (that is why we need communications). We will still compute the update of the edge cells on the local grid. We can write the algorithm in the following way, which is executed on each subgrid:
\begin{itemize}
\item Get the edges of the neighbours $G_{i\pm1, j\pm1}$ (and send the edges of the local grid to the neighbours requesting it).
\item For each cell in the local grid, compute the updated value of the cell, and put it in a temporary grid $T_{i, j}$
\item Assign $T_{i,j}$ to $G_{i,j}$
\end{itemize}

\paragraph{Question 2c)}
Let $d$ be the cost of the application of $\delta$. To update a local grid, there are $\frac{N}{p}\frac{M}{q}$ calls to $\delta$. We also need to send $2\frac{N}{p} + 2\frac{M}{q} + 4$ cases, and receive the same number. Thus, the communications costs are ???. Depends on whether we should really consider the $+4$. In that case, how to communicate with $G_{i-1, j-1}$ for example ?\\
NB : we can do in parallel the computations of the inner part of the grid, and the communications with other nodes. And then update the edges.

\paragraph{Question 2d)} ? A bit less in the communicatin costs.

\section{Laplace operator and convolution}

\paragraph{Question 3a)} We consider the following cellular automaton: $\mathcal{A} = (2, \mathbb{Z}^2, 1, \delta)$, with:

\begin{eqnarray*}
\delta : \left \{ \begin{array}{c c l }
  \mathbb{Z}^2 & \longrightarrow & \mathbb{Z}^2 \\
  u^{t+1}_{i,j}\ \pmb{,} \ u^{_{'} t+1}_{i,j} & \longmapsto & u^{t}_{i,j} + u^{_{'} t}_{i,j}\ \pmb{,} \ u^{_{'} t}_{i,j} + v^2 \cdot (u^t_{i+1,j} + u^t_{i-1, j} + u^t_{i, j+1} + u^t_{i, j-1} - 4 u^t_{i, j})\\
  \end{array} \right .
\end{eqnarray*}

NB: $\mathbb{Z}$ should be replaced by the set of C doubles...

\section{Performance evaluation}

\paragraph{Question 4a)} Simgrid or real computations ?

\paragraph{Question 4b)} Good question. We could see what wins in the end between communication and computation.


\section{Walls}

\paragraph{Question 5a)} We should transform $\mathcal{Q}$ into $\{0, 1\} \times \mathbb{Z}^2$, to know if a cell is active or is a wall (0 meaning that the cell is a wall). We define two auxiliary functions f and f', and then update $\delta$: 


\begin{eqnarray*}
f : \left \{ \begin{array}{c c l }
  \mathcal{Q} & \longrightarrow & \mathbb{Z} \\
  s \ \pmb{,} \ u\ \pmb{,} \ u' & \longmapsto & \mathbbm{1}_{\mathbb{N}^*}(s) \cdot u
  \end{array} \right .
\end{eqnarray*}


\begin{eqnarray*}
f' : \left \{ \begin{array}{c c l }
  \mathcal{Q} & \longrightarrow & \mathbb{Z} \\
  s \ \pmb{,} \ u\ \pmb{,} \ u' & \longmapsto & \mathbbm{1}_{\mathbb{N}^*}(s) \cdot u'
  \end{array} \right .
\end{eqnarray*}


\begin{eqnarray*}
\delta : \left \{ \begin{array}{c c l }
  \mathcal{Q} & \longrightarrow & \mathcal{Q} \\
  x_{i,j}^t = (s \ \pmb{,} \ u^{t+1}_{i,j}\ \pmb{,} \ u^{_{'} t+1}_{i,j}) & \longmapsto &
  \begin{cases}s, 0, 0\mbox{ if s = 0}\\
    s, f(x^{t}_{i,j}) + f'(x^t_{i,j})\ \pmb{,} f'(x^t_{i,j}) + v^2 \cdot (f(x^t_{i+1,j}) \\+ f(x^t_{i-1, j}) + f(x^t_{i, j+1}) + f(x^t_{i, j-1}) - 4 f(x^t_{i, j})) 
    \end{cases}\\
  \end{array} \right .
\end{eqnarray*}

\section{Sensors}

\paragraph{Question 6a)} We suppose that sensors are never walls (it would not be interesting at all, as the record variable would always be 0). We also suppose that we record only $u_{i,j}^2$ and not $q^2$ for some $q \in \mathcal{Q}$. So, we just need to change $\mathcal{Q}$ into $\{0,1,2\} \times \mathbb{Z}^2 \times \mathbb{N}$, where 0 means a wall, 1 means a normal cell, and 2 means a sensor. We need to update $\delta$:

\begin{eqnarray*}
\delta : \left \{ \begin{array}{c c l }
  \mathcal{Q} & \longrightarrow & \mathcal{Q} \\
  x_{i,j}^t = (s \ \pmb{,} \ u^{t+1}_{i,j}\ \pmb{,} \ u^{_{'} t+1}_{i,j} \ \pmb{,} \ c) & \longmapsto &
  \begin{cases}s, 0, 0\mbox{ if s = 0}\\
    s, f(x^{t}_{i,j}) + f'(x^t_{i,j})\ \pmb{,} f'(x^t_{i,j}) + v^2 \cdot (f(x^t_{i+1,j}) \\+ f(x^t_{i-1, j}) + f(x^t_{i, j+1}) + f(x^t_{i, j-1}) - 4 f(x^t_{i, j})),\\
    \begin{cases} 0 \mbox{ if }s \leq 1\\
      c + (f(x^{t}_{i,j}) + f'(x^t_{i,j}))^2 \mbox{ if s = 2} \\
    \end{cases}
  \end{cases}
  \end{array} \right .
\end{eqnarray*}


\paragraph{Question 7a)} We need to change $\mathcal{Q}$ into $\{0,1,2\} \times \mathbb{Z}^2 \times \mathbb{N} \times \mathbb{R}$, the last argument being the local speed. In $\delta$, we just update $v$ into $v_{i,j}$.

%% TODO : Un peu la flemme de refaire tout ce delta moche juste pour la vitesse

\section{(Experimental) Science}

\paragraph{Question 8a)} We need to check if this is close to reality or not. We could try to see if the propagation is correct, and then try to observe more complexe phenomenon, like interferences.

\end{document}
